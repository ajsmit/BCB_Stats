% Options for packages loaded elsewhere
\PassOptionsToPackage{unicode}{hyperref}
\PassOptionsToPackage{hyphens}{url}
\PassOptionsToPackage{dvipsnames,svgnames,x11names}{xcolor}
%
\documentclass[
  letterpaper,
  DIV=11,
  numbers=noendperiod]{scrartcl}

\usepackage{amsmath,amssymb}
\usepackage{iftex}
\ifPDFTeX
  \usepackage[T1]{fontenc}
  \usepackage[utf8]{inputenc}
  \usepackage{textcomp} % provide euro and other symbols
\else % if luatex or xetex
  \usepackage{unicode-math}
  \defaultfontfeatures{Scale=MatchLowercase}
  \defaultfontfeatures[\rmfamily]{Ligatures=TeX,Scale=1}
\fi
\usepackage{lmodern}
\ifPDFTeX\else  
    % xetex/luatex font selection
\fi
% Use upquote if available, for straight quotes in verbatim environments
\IfFileExists{upquote.sty}{\usepackage{upquote}}{}
\IfFileExists{microtype.sty}{% use microtype if available
  \usepackage[]{microtype}
  \UseMicrotypeSet[protrusion]{basicmath} % disable protrusion for tt fonts
}{}
\makeatletter
\@ifundefined{KOMAClassName}{% if non-KOMA class
  \IfFileExists{parskip.sty}{%
    \usepackage{parskip}
  }{% else
    \setlength{\parindent}{0pt}
    \setlength{\parskip}{6pt plus 2pt minus 1pt}}
}{% if KOMA class
  \KOMAoptions{parskip=half}}
\makeatother
\usepackage{xcolor}
\setlength{\emergencystretch}{3em} % prevent overfull lines
\setcounter{secnumdepth}{-\maxdimen} % remove section numbering
% Make \paragraph and \subparagraph free-standing
\ifx\paragraph\undefined\else
  \let\oldparagraph\paragraph
  \renewcommand{\paragraph}[1]{\oldparagraph{#1}\mbox{}}
\fi
\ifx\subparagraph\undefined\else
  \let\oldsubparagraph\subparagraph
  \renewcommand{\subparagraph}[1]{\oldsubparagraph{#1}\mbox{}}
\fi


\providecommand{\tightlist}{%
  \setlength{\itemsep}{0pt}\setlength{\parskip}{0pt}}\usepackage{longtable,booktabs,array}
\usepackage{calc} % for calculating minipage widths
% Correct order of tables after \paragraph or \subparagraph
\usepackage{etoolbox}
\makeatletter
\patchcmd\longtable{\par}{\if@noskipsec\mbox{}\fi\par}{}{}
\makeatother
% Allow footnotes in longtable head/foot
\IfFileExists{footnotehyper.sty}{\usepackage{footnotehyper}}{\usepackage{footnote}}
\makesavenoteenv{longtable}
\usepackage{graphicx}
\makeatletter
\def\maxwidth{\ifdim\Gin@nat@width>\linewidth\linewidth\else\Gin@nat@width\fi}
\def\maxheight{\ifdim\Gin@nat@height>\textheight\textheight\else\Gin@nat@height\fi}
\makeatother
% Scale images if necessary, so that they will not overflow the page
% margins by default, and it is still possible to overwrite the defaults
% using explicit options in \includegraphics[width, height, ...]{}
\setkeys{Gin}{width=\maxwidth,height=\maxheight,keepaspectratio}
% Set default figure placement to htbp
\makeatletter
\def\fps@figure{htbp}
\makeatother

\KOMAoption{captions}{tableheading}
\makeatletter
\@ifpackageloaded{tcolorbox}{}{\usepackage[skins,breakable]{tcolorbox}}
\@ifpackageloaded{fontawesome5}{}{\usepackage{fontawesome5}}
\definecolor{quarto-callout-color}{HTML}{909090}
\definecolor{quarto-callout-note-color}{HTML}{0758E5}
\definecolor{quarto-callout-important-color}{HTML}{CC1914}
\definecolor{quarto-callout-warning-color}{HTML}{EB9113}
\definecolor{quarto-callout-tip-color}{HTML}{00A047}
\definecolor{quarto-callout-caution-color}{HTML}{FC5300}
\definecolor{quarto-callout-color-frame}{HTML}{acacac}
\definecolor{quarto-callout-note-color-frame}{HTML}{4582ec}
\definecolor{quarto-callout-important-color-frame}{HTML}{d9534f}
\definecolor{quarto-callout-warning-color-frame}{HTML}{f0ad4e}
\definecolor{quarto-callout-tip-color-frame}{HTML}{02b875}
\definecolor{quarto-callout-caution-color-frame}{HTML}{fd7e14}
\makeatother
\makeatletter
\@ifpackageloaded{caption}{}{\usepackage{caption}}
\AtBeginDocument{%
\ifdefined\contentsname
  \renewcommand*\contentsname{Table of contents}
\else
  \newcommand\contentsname{Table of contents}
\fi
\ifdefined\listfigurename
  \renewcommand*\listfigurename{List of Figures}
\else
  \newcommand\listfigurename{List of Figures}
\fi
\ifdefined\listtablename
  \renewcommand*\listtablename{List of Tables}
\else
  \newcommand\listtablename{List of Tables}
\fi
\ifdefined\figurename
  \renewcommand*\figurename{Figure}
\else
  \newcommand\figurename{Figure}
\fi
\ifdefined\tablename
  \renewcommand*\tablename{Table}
\else
  \newcommand\tablename{Table}
\fi
}
\@ifpackageloaded{float}{}{\usepackage{float}}
\floatstyle{ruled}
\@ifundefined{c@chapter}{\newfloat{codelisting}{h}{lop}}{\newfloat{codelisting}{h}{lop}[chapter]}
\floatname{codelisting}{Listing}
\newcommand*\listoflistings{\listof{codelisting}{List of Listings}}
\makeatother
\makeatletter
\makeatother
\makeatletter
\@ifpackageloaded{caption}{}{\usepackage{caption}}
\@ifpackageloaded{subcaption}{}{\usepackage{subcaption}}
\makeatother
\ifLuaTeX
  \usepackage{selnolig}  % disable illegal ligatures
\fi
\usepackage{bookmark}

\IfFileExists{xurl.sty}{\usepackage{xurl}}{} % add URL line breaks if available
\urlstyle{same} % disable monospaced font for URLs
\hypersetup{
  pdftitle={Wind stress and wind stress curl},
  colorlinks=true,
  linkcolor={blue},
  filecolor={Maroon},
  citecolor={Blue},
  urlcolor={Blue},
  pdfcreator={LaTeX via pandoc}}

\title{Wind stress and wind stress curl}
\author{}
\date{}

\begin{document}
\maketitle

\subsection{Relative Vorticity}\label{relative-vorticity}

Relative vorticity is a measure of the rotation of fluid parcels
relative to the Earth's surface. It is a key concept in both meteorology
and oceanography for describing the rotation of air or water masses. For
a two-dimensional horizontal flow, the relative vorticity \(\zeta\) is
defined as:

\[
\zeta = \frac{\partial v}{\partial x} - \frac{\partial u}{\partial y}
\]

where:

\begin{itemize}
\tightlist
\item
  \(u\) and \(v\) are the velocity components of the fluid in the x
  (east-west) and y (north-south) directions, respectively.
\item
  \(\frac{\partial v}{\partial x}\) and
  \(\frac{\partial u}{\partial y}\) are the partial derivatives of these
  components with respect to the spatial coordinates.
\end{itemize}

Relative vorticity indicates how much the fluid is spinning or rotating
at a point, excluding the rotation of the Earth (which would be included
in absolute vorticity).

\subsection{Wind Stress}\label{wind-stress}

Wind stress (\(\tau\)) is the horizontal force exerted by wind on the
ocean's surface, acting as a drag force per unit area. It is essentially
the friction between the moving air and the water, driving ocean
currents and influencing various marine processes.

We can use the following equation to calculate surface wind stress for
each grid point (lon, lat, time) wind value in a dataframe:

\[\tau = \rho_{a}C_{d}U^{2}\]

where:

\begin{itemize}
\tightlist
\item
  \(\rho_{a}\) is the air density (1.225 kg/m³),
\item
  \(C_{d}\) is the drag coefficient (0.0012), a dimensionless quantity
  that depends on wind speed and sea state, and
\item
  \(U\) is the wind speed (m/s) at the reference height.
\end{itemize}

The zonal (east-west) and meridional (north-south) wind stress
components are derived from the wind speed at a reference height
(usually 10 meters above the sea surface) and a drag coefficient. Here
are the equations:

\[\tau_{x} = \tau \cos(\theta)\]

\[\tau_{y} = \tau \sin(\theta)\]

where:

\begin{itemize}
\tightlist
\item
  \(\theta\) is the angle of the wind direction relative to the east (0°
  for east winds, 90° for north winds, etc.),
\item
  \(\tau_x\) is the zonal wind stress (force per unit area in the
  east-west direction), and
\item
  \(\tau_y\) is the meridional wind stress (force per unit area in the
  north-south direction)
\end{itemize}

\begin{tcolorbox}[enhanced jigsaw, titlerule=0mm, rightrule=.15mm, opacityback=0, opacitybacktitle=0.6, colback=white, coltitle=black, arc=.35mm, left=2mm, leftrule=.75mm, toptitle=1mm, breakable, colframe=quarto-callout-note-color-frame, bottomtitle=1mm, toprule=.15mm, colbacktitle=quarto-callout-note-color!10!white, title=\textcolor{quarto-callout-note-color}{\faInfo}\hspace{0.5em}{Sine and cosine functions}, bottomrule=.15mm]

In the above formulations of zonal and meridional wind stress
components, \(\tau_x\) uses the cosine function, and \(\tau_y\) uses the
sine function. This is for wind direction defined relative to the east.
If the wind direction is defined relative to the north, the sine
function would be used for \(\tau_x\), and the cosine function would be
used for \(\tau_y\).

\end{tcolorbox}

\subsection{Wind stress curl}\label{wind-stress-curl}

`Curl' is a measure of the rotation of a vector field; therefore, wind
stress curl is a measure of the rotation of the wind stress field. It is
defined as the vertical component of the curl of the wind stress
vector---since it is a vector quantity it has both magnitude and
direction.

Wind stress curl is computed using the following basic relation:

\[\nabla \times \tau = \frac{\partial \tau_{y}}{\partial x} - \frac{\partial \tau_{x}}{\partial y}\]

where:

\begin{itemize}
\tightlist
\item
  \(\nabla \times \tau\) is the wind stress curl,
\item
  \(\frac{\partial \tau_{y}}{\partial x}\) is the derivative of the
  meridional wind stress with respect to the zonal direction, and
\item
  \(\frac{\partial \tau_{x}}{\partial y}\) is the derivative of the
  zonal wind stress with respect to the meridional direction.
\end{itemize}

The direction of the wind stress curl is parallel to the \(z\)-axis. By
convention, it is typically perpendicular to Earth's surface. Wind
stress curl is influenced by the Coriolis effect, which causes fluids
(air and water) to deflect to the right in the Northern Hemisphere and
to the left in the Southern Hemisphere due to the Earth's rotation.

The relationship between wind stress curl and circulation patterns is as
follows:

\begin{itemize}
\tightlist
\item
  \textbf{Positive Wind Stress Curl:} Indicates cyclonic
  (counter-clockwise) circulation in the Northern Hemisphere and
  anticyclonic (clockwise) circulation in the Southern Hemisphere.
\item
  \textbf{Negative Wind Stress Curl:} Indicates anticyclonic (clockwise)
  circulation in the Northern Hemisphere and cyclonic
  (counter-clockwise) circulation in the Southern Hemisphere.
\end{itemize}

Use the right-hand rule to visualise this relationship. If you curl the
fingers of your right hand in the direction of the wind stress vectors,
your thumb will point in the direction of the wind stress curl vector.
If the wind stress curl vector points upwards (positive), the
circulation is counter-clockwise in the Northern Hemisphere and
clockwise in the Southern Hemisphere. If the wind stress curl vector
points downwards (negative), the circulation is clockwise in the
Northern Hemisphere and counter-clockwise in the Southern Hemisphere.

Therefore, cyclones (low-pressure systems) in the Northern Hemisphere
are associated with positive wind stress curl, while cyclones in the
Southern Hemisphere are associated with negative wind stress curl.
Conversely, anticyclones (high-pressure systems) in the Northern
Hemisphere are associated with negative wind stress curl, while
anticyclones in the Southern Hemisphere are associated with positive
wind stress curl.

\subsubsection{Implications of a Very Negative Wind Stress Curl in the
Southern
Hemisphere}\label{implications-of-a-very-negative-wind-stress-curl-in-the-southern-hemisphere}

\begin{enumerate}
\def\labelenumi{\arabic{enumi}.}
\item
  \textbf{Oceanographic Effects}:

  \begin{itemize}
  \tightlist
  \item
    \textbf{Upwelling} and \textbf{Downwelling}:

    \begin{itemize}
    \tightlist
    \item
      Wind stress curl is closely related to Ekman pumping.
    \item
      Upwelling occurs near the coast when the wind stress has an
      equatorward alongshore component or in the case of nearshore
      negative (in the Southern Hemisphere) wind stress curl.
    \end{itemize}
  \item
    \textbf{Gyre Circulation}:

    \begin{itemize}
    \tightlist
    \item
      In large-scale ocean circulation, negative wind stress curl is
      associated with anticyclonic gyres (e.g., subtropical gyres).
    \item
      These gyres are characterized by downwelling in the center, which
      results in a thicker and warmer water column.
    \end{itemize}
  \end{itemize}
\item
  \textbf{Ecological Impact}:

  \begin{itemize}
  \tightlist
  \item
    Downwelling regions typically have lower nutrient concentrations at
    the surface because nutrients are pushed downward with the sinking
    water.
  \item
    This can lead to reduced primary productivity and lower
    phytoplankton biomass, affecting the entire marine food web.
  \item
    In regions such as the Southern Ocean around the Antarctic
    Circumpolar Current, variations in wind stress curl can
    significantly impact the distribution of nutrients and biological
    productivity.
  \end{itemize}
\item
  \textbf{Climate and Weather}:

  \begin{itemize}
  \tightlist
  \item
    Areas of persistent negative wind stress curl can influence local
    and regional climate patterns by affecting sea surface temperatures
    and heat distribution in the ocean.
  \item
    These areas may also impact weather patterns, such as the formation
    and intensity of cyclones and anticyclones.
  \item
    In the Southern Hemisphere, this can influence the climate dynamics
    of nearby landmasses, including southern parts of continents like
    South America, Africa, and Australia.
  \end{itemize}
\end{enumerate}

A very negative wind stress curl in the Southern Hemisphere indicates a
strong clockwise (anticyclonic) rotation of the wind stress field,
leading to downwelling and associated oceanographic and ecological
effects. This influences large-scale ocean circulation, nutrient
distribution, primary productivity, and potentially local and regional
climate and weather patterns. Understanding and monitoring wind stress
curl is crucial for predicting ocean behavior and its broader
environmental impacts, especially in the context of the Southern
Hemisphere's unique oceanic and atmospheric dynamics.



\end{document}
